\documentclass{article}
\usepackage{algorithm}
\usepackage{algorithmic}
\usepackage{pgfplots}
\pgfplotsset{compat=1.8}
\usepackage{filecontents}
\usepackage{hyperref}
\hypersetup{
	 colorlinks   = true,
     citecolor    = black,
     linkcolor    = black,
     urlcolor     = black
}
\usepackage{courier}
\usepackage{graphicx}
\usepackage{float}
\usepackage{xeCJK}
\setcounter{tocdepth}{2}

\makeatletter
\newcommand{\algorithmicbreak}{\textbf{break}}
\newcommand{\BREAK}{\STATE \algorithmicbreak}
\makeatother

\begin{filecontents*}{naive.dat}
size naive
1e05 43
1e06 94
1e07 696
\end{filecontents*}

\begin{filecontents*}{tscube.dat}
size tscube
1e05 129
1e06 180
1e07 563
1e08 4128
\end{filecontents*}

\begin{document}
\bibliographystyle{acm}

\title{Course Project - Principles of Database Systems \\ Hadoop branch}
\author{张秋怡 12330402 \\ \href{mailto:joyeec9h3@gmail.com}{joyeec9h3@gmail.com}} 
\date{\today}
\maketitle

\tableofcontents
\section{Team Members}

\begin{table}[H]
\centering
\begin{tabular}{l l l}
Name              & Student number & Mail \\
\hline
张秋怡 & 12330402 &  \href{mailto:joyeec9h3@gmail.com}{joyeec9h3@gmail.com}  \\
郑沛翼 & 12330418 &  \href{mailto:187840@qq.com}{187840@qq.com}  \\
郑安恺 & 12330415 &  \href{mailto:573476807@qq.com}{573476807@qq.com}  \\
郑穗展 & 12330402 &  \href{mailto:zhengszh3@gmail.com}{zhengszh3@gmail.com}  \\
郑恺培 & 12330417 &  \href{mailto:753494474@qq.com}{753494474@qq.com}
\end{tabular}
\end{table}

\section{System Design}

\subsection{Overview}

\subsection{Naive Cube}

\subsection{TSCube}

\section{Implementation}

\paragraph{}

\subsection{Estimate}

\subsection{Materialize}

During the implementation, we discover that in batches where there are two hierarchies involved, the field in the first hierarchy would disturb the cubing process of the second one (illustrated in Table~\ref{table:interrupt}). Therefore, we decide to take an approach more simliar to the BUC algorithm mentioned in \cite{beyer1999bottom} than the top-down cubing algorithm discussed in \cite{agarwal1996computation} and \cite{zhao1997array}. Instead of scanning the fields from the right, updating the previous set and output the measurement of current field on changes, we scan the fields from the left, updating the sets as soon as a new tuple is scanned, and output the measurement of following fields on changes. This approach would not cost more memory than the original top-down cubing algorithm, since th number and sizes of the sets we need to maintain will not change greatly. The most significant change lies in the number of insertion into these sets. The pseudocode of this algorithm is explained in Algorithm~\ref{alg:cubeal}:

\begin{table}[H]
\centering
\begin{tabular}{l l l l l l}
country & state & city & topic & category & product \\
\hline
1 & 2 & 10 & 1 & 20 & 5 \\
1 & 2 & 13 & 1 & 20 & 5 \\
1 & 2 & 14 & 1 & 20 & 5

\end{tabular}
\caption{In this example, the topics of the records don't change while the city has changed, so the uids in the $Set_{topic=1}$ will all be merged into the $Set_{city=10}$, resulting in wrong measurements.}
\label{table:interrupt}
\end{table}

\begin{algorithm}[H]
\centering
\caption{The cubing algorithm}
\label{alg:cubeal}
\begin{algorithmic}[1]  
\FOR{each K-V pair}
\STATE $B \Leftarrow$ the batch this K-V pair belongs to
\STATE $dim \Leftarrow$ The starting dimension of B
\STATE $numDims \Leftarrow$ The total number of dimensions of B
\FOR{$i=dim$; $i<numDims$; $i++$ }
\IF{The field in $i$ changes}
\STATE Output the measure of this and each subsequent dimensions
\STATE Update the sets of this and each subsequent dimensions
\BREAK
\ELSE
\STATE Add the uid into $Set[i]$
\ENDIF
\ENDFOR
\ENDFOR   
\end{algorithmic}  
\end{algorithm}

\subsection{Postprocess}

\section{Experimental results}

\subsection{Environment settings}

\paragraph{}
We perform the experiment on a cluster made up of our laptops. There are 4 nodes in total, 3 of them are slaves. The hardware environment is listed in Table~\ref{table:env} Each of them runs Hadoop-1.0.4 and Python 2.7+.

\begin{table}[H]
\centering
\begin{tabular}{l l l l l}
Role & OS & Memory & CPU & Disk \\
\hline
Master & Ubuntu 12.04 & 3848M & Intel i7-3517U & 256GB \\
Slave & Ubuntu 13.04  & 7882M & Intel i7-3610QM & 500GB \\
Slave & Ubuntu 13.04 & 1978M & Intel Pentium B950 & 500GB \\
Slave & Ubuntu 13.04 & 3547M & Intel i3-2330MB & 500GB
\end{tabular}

\caption{Hardware environment}
\label{table:env}
\end{table}

\subsection{Experimental results on cluster}

\begin{figure}[H]

\begin{tikzpicture}
\begin{loglogaxis}[
title={Performance of Naive cube and TSCube},
xlabel={Data Size},
ylabel={Time(s)},
legend pos=outer north east
]
\addplot table[x=size,y=naive] {naive.dat};\addlegendentry{Naive}
\addplot table[x=size,y=tscube] {tscube.dat};\addlegendentry{TSCube}
\end{loglogaxis}
\end{tikzpicture}

\caption{Performance of Naive cube and TSCube}
\label{fig:performance}
\end{figure}

\section{Reflection}

\bibliography{tscube}

\end{document}